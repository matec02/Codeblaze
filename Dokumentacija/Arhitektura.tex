\chapter{Arhitektura i dizajn sustava}

\subsection{Opis arhitekture}

Detaljnom razradom cilja projektnog zadatka, u kojem je fokus izrada aplikacije za iznajmljivanje električnih romobila, definirali smo razinu klijenta, razinu web-aplikacije te sloj pristupa podatcima kao osnovne razine naše aplikacije.
\newline
\subsubsection{Razina klijenta}

Razina klijenta predstavlja korisničko sučelje web-aplikacije koje korisnici vide i s kojim interagiraju. U razvoju projekta korišten je React, odnosno JavaScript knjižnica za izradu korisničkog sučelja. Projekt je organiziran u komponente koje predstavljaju određene dijelove korisničkog sučelja. Korišten je virtualni DOM (Document Object Model), kojim se ubrzava proces ažuriranja promjena korisničkog sučelja u svrhu poboljšavanja performansi web-aplikacije.
\newline
\subsubsection{Razina web-aplikacije}

Sloj web-aplikacije je odgovoran za obradu zahtjeva korisnika, poslovnu logiku i komunikaciju s bazom podataka. U razvoju projekta korišten je okvir za razvoj web aplikacija Spring Boot u programskom jeziku Java. U Spring Bootu, kontroleri su odgovorni za obradu HTTP zahtjeva i usmjeravanje na odgovarajuće servise za obradu zahtjeva. Obradu podataka, validaciju te logiku obavljaju servisi, dok modeli predstavljaju strukturu podataka koja se koristi za komunkaciju s bazom podataka te prenošenje podataka između kontrolera i servisa.
\newline

\subsubsection{Sloj pristupa podatcima}

Sloj pristupa podacima je odgovoran za komunikaciju s bazom podataka i pristupanje podacima. Građen je od entiteta s vlastitim atributima koji predstavljaju modele podataka koji odgovaraju tablicama u bazi podataka.
\newline
Sinteza ovih slojeva - korisničkog sučelja na razini Reacta, web-aplikacijskog sloja u Spring Bootu i sloja pristupa podacima u Spring Bootu - stvara temelj za razvoj visoko skalabilnih i funkcionalnih web-aplikacija. Korisnici ostvaruju interakciju s aplikacijom preko intuitivnog React korisničkog sučelja, dok Spring Boot preuzima odgovornost za obradu njihovih zahtjeva i poslovne logike. Istovremeno, sloj pristupa podacima omogućuje efikasnu komunikaciju s bazom podataka, omogućujući pohranu i dohvat podataka s pouzdanošću i učinkovitošću.
\newline
\subsection{MVC arhitektura}
Model-View-Controller (MVC) je arhitekturni obrazac koji se koristi za organizaciju komponenti u softverskim aplikacijama, posebno u razvoju web-aplikacija. Osnovna svrha MVC-a je odvajanje različitih aspekata aplikacije kako bi se omogu- ćila bolja organizacija, održavanje i skalabilnost. Sastoji se od tri glavne komponente:
\begin{itemize}
	\item\textbf{Model} - predstavlja sloj koji je odgovoran za obradu podataka i poslovnu logiku aplikacije te sadrži podatke i pravila za njihovu obradu.
	\item\textbf{View} - predstavlja sloj koji se odnosi na korisničko sučelje aplikacije i odgovoran je za prikazivanje podataka korisnicima. Ne obavlja nikakvu poslovnu logiku, samo prikazuje podatke koji mu se dostave iz modela.
	\item\textbf{Kontroler} - posrednik između Model i View komponenti. Prima korisničke zahtjeve, obrađuje ih te komunicira s Modelom radi dohvaćanja ili ažuriranja podataka. Također, odlučuje koji View treba biti prikazan korisniku na temelju podataka iz Modela te korisničkih zahtjeva, upravlja tokom aplikacije te sadrži logiku za validaciju, autorizaciju i upravljanje sesijama.
\end{itemize}

MVC arhitektura omogućuje precizno razgraničenje odgovornosti unutar aplikacije. Ovo strukturalno odvajanje olakšava razvoj aplikacije, čini ju lakšom za održavanje i omogućava efikasnije testiranje. Svaka od tri glavne komponente - Model, View i Controller - može se ponovno koristiti na različitim dijelovima aplikacije. To potiče efikasnost razvoja jer se već razvijeni dijelovi aplikacije mogu lako iskoristiti u drugim kontekstima. MVC omogućuje skalabilnost aplikacije jer se jasno razdvajaju različiti aspekti. Novi dijelovi funkcionalnosti mogu se dodavati bez narušavanja postojeće arhitekture, što omogućava aplikaciji da raste i prilagodi se promjenama.

\begin{figure} [H]

	\includegraphics[width=1\linewidth]{slike/mvc.png}
	\centering
	\caption{Prikaz MVC obrasca}
	\label{fig:Prikaz MVC obrasca}
\end{figure}

\section{Baza podataka}

U kontekstu našeg sustava, baza podataka igra ključnu ulogu, pružajući strukturiranu platformu za modeliranje stvarnog svijeta. Temeljni građevni blok ove baze je relacija, odnosno tablica koja je jasno definirana svojim imenom i skupom atributa. Glavna svrha baze podataka je olakšati brzu i jednostavnu pohranu, promjenu te izvlačenje podataka kako bi se omogućila daljnja analiza i obrada. Unutar baze podataka za našu aplikaciju, identificiramo nekoliko ključnih entiteta:

	\begin{itemize}
		\item 	\textit{User}
		\item 	\textit{PrivacySettings}
		\item 	\textit{Document}
		\item 	\textit{Scooter}
		\item 	\textit{Listing}
		\item 	\textit{Review}
		\item 	\textit{Transaction}
		\item 	\textit{ChatSession}
		\item 	\textit{Message}
		\item 	\textit{ImageChangeRequest}
	\end{itemize}


\subsection{User}


Ovaj entitet sadrzava sve važne informacije o korisniku aplikacije. Sadrži atribute: userId, nickname, firstName, lastName, cardNumber, email, phoneNumber, password, role te status. Ovaj entitet ima \textit{One-to-one} vezu s entitetom Preferences preko atributa userId, \textit{One-to-one} vezu s entitetom PrivacySettings preko atributa userId, \textit{One-to-one} vezu s entitetom SocialMedia preko atributa userId, \textit{One-to-one} vezu s entitetom Document preko atributa userId, \textit{One-to-many} vezu s entitetom Scooter preko atributa userId, \textit{One-to-many} vezu s entitetom Listing preko atributa renterUsername, \textit{One-to-one} vezu s entitetom Review preko atributa reviewerUsername, \textit{One-to-one} vezu s entitetom Review preko atributa renterUsername, \textit{One-to-many} vezu s entitetom ChatSessions preko atributa user1 ili atributa user2, \textit{Many-to-one} vezu s entitetom Messages preko atributa senderUsername, \textit{One-to-many} vezu s entitetom ImageChangeRequest preko atributa requesterId te \textit{One-to-many} vezu s entitetom Notification preko atributa userId, requestingUser te decisionAdmin.


\begin{longtblr}[
	label=none,
	entry=none
]{
	width = \textwidth,
	colspec={|X[10,l]|X[7, l]|X[15, l]|},
	rowhead = 1,
} %definicija širine tablice, širine stupaca, poravnanje i broja redaka naslova tablice
	\hline \SetCell[c=3]{c}{\textbf{User}}	 \\ \hline[3pt]
	\SetCell{LightGreen}userId & INT	&  jedinstveni identifikator korisnika	\\ \hline
	nickname	& VARCHAR &  jedinstveni nadimak korisnika  	\\ \hline
	firstName & VARCHAR &  ime korisnika  \\ \hline
	lastName & VARCHAR	& prezime korisnika
	\\ \hline
	cardNumber	& INT &   broj kartice korisnika	\\ \hline
	email	& VARCHAR &    jedinstvena email adresa korisnika	\\ \hline
	phoneNumber	& INT &   jedinstveni kontakt broj korisnika 	\\ \hline
	password	& VARCHAR & zaporka za prijavu korisnika   	\\ \hline
	role	& UserRole &  uloga korisnika (unregistered, registered, renter, admin) 	\\ \hline
	status	& UserStatus & status korisnika (pending, rejected, accepted, deleted, blocked) 	\\ \hline
\end{longtblr}


\subsection{Preferences}


Ovaj entitet sadrzava sve važne informacije o preferencama korisnika aplikacije. Sadrži atribute: userId, language i darkMode. Ovaj entitet ima \textit{One-to-one} vezu s entitetom User preko atributa userId.


\begin{longtblr}[
	label=none,
	entry=none
]{
	width = \textwidth,
	colspec={|X[10,l]|X[7, l]|X[15, l]|},
	rowhead = 1,
} %definicija širine tablice, širine stupaca, poravnanje i broja redaka naslova tablice
	\hline \SetCell[c=3]{c}{\textbf{Preferences}}	 \\ \hline[3pt]
	\SetCell{LightGreen}userId & INT	&  jedinstveni identifikator korisnika (Ref: User.userId) \\ \hline
	language	& UserLanguage & jezik korisnika   	\\ \hline
	darkMode & BOOLEAN &  omogućen dark mode \\ \hline
\end{longtblr}

\subsection{Social Media}


Ovaj entitet sadrzava sve važne informacije o socijalnim mrežama korisnika aplikacije. Sadrži atribute: userId, instagram, facebook, google i tikTok. Ovaj entitet ima \textit{One-to-one} vezu s entitetom User preko atributa userId.


\begin{longtblr}[
	label=none,
	entry=none
]{
	width = \textwidth,
	colspec={|X[10,l]|X[7, l]|X[15, l]|},
	rowhead = 1,
} %definicija širine tablice, širine stupaca, poravnanje i broja redaka naslova tablice
	\hline \SetCell[c=3]{c}{\textbf{Social Media}}	 \\ \hline[3pt]
	\SetCell{LightGreen}userId & INT	&  jedinstveni identifikator korisnika (Ref: User.userId)	\\ \hline
	instagram	& VARCHAR &   	instagram account korisnika \\ \hline
	facebook & VARCHAR & facebook account korisnika  \\ \hline
	google & VARCHAR	&  google account korisnika		\\ \hline
	tikTok	& VARCHAR &  tikTok account korisnika 	\\ \hline
\end{longtblr}

\subsection{PrivacySettings}


Ovaj entitet sadrzava sve važne informacije o postavkama privatnosti korisnika aplikacije. Sadrži atribute: userId, isFirstNameVisible, isLastNameVisible, isNicknameVisible i isEmailVisible. Ovaj entitet ima \textit{One-to-one} vezu s entitetom User preko atributa userId.


\begin{longtblr}[
	label=none,
	entry=none
]{
	width = \textwidth,
	colspec={|X[10,l]|X[7, l]|X[15, l]|},
	rowhead = 1,
} %definicija širine tablice, širine stupaca, poravnanje i broja redaka naslova tablice
	\hline \SetCell[c=3]{c}{\textbf{Privacy Settings}}	 \\ \hline[3pt]
	\SetCell{LightGreen}userId & INT	&  	 jedinstveni identifikator korisnika (Ref: User.userId)	\\ \hline
	isFirstNameVisible	& BOOLEAN &   omogućena vidljivost imena korisnika	\\ \hline
	isLastNameVisible & BOOLEAN &   omogućena vidljivost prezimena korisnika\\ \hline
	isNicknameVisible & BOOLEAN	&  		omogućena vidljivost nadimka korisnika\\ \hline
	isEmailVisible & BOOLEAN	&  		omogućena vidljivost emaila korisnika\\ \hline
\end{longtblr}

\subsection{Document}


Ovaj entitet sadrzava sve važne informacije o dokumentima korisnika aplikacije. Sadrži atribute: userId, pathCriminalRecord, pathIdentification i status. Ovaj entitet ima \textit{One-to-one} vezu s entitetom User preko atributa userId.


\begin{longtblr}[
	label=none,
	entry=none
]{
	width = \textwidth,
	colspec={|X[10,l]|X[7, l]|X[15, l]|},
	rowhead = 1,
} %definicija širine tablice, širine stupaca, poravnanje i broja redaka naslova tablice
	\hline \SetCell[c=3]{c}{\textbf{Document}}	 \\ \hline[3pt]
	\SetCell{LightGreen}userId & INT	&  jedinstveni identifikator korisnika (Ref: User.userId)	\\ \hline
	pathCriminalRecord	& VARCHAR &  url kaznene evidencije	\\ \hline
	pathIdentification & VARCHAR &  url identifikacijskog dokumenta \\ \hline
	status & DocumentStatus	& status dokumenta (pending, approved, rejected) 		\\ \hline
\end{longtblr}

\subsection{Scooter}


Ovaj entitet sadrzava sve važne informacije o pojedinom romobilu. Sadrži atribute: scooterId, manufacturer, model, batteryCapacity, maxSpeed, imagePath, maxRange, yearOfManufacture, additionalInformation, userId, deleted i availability. Ovaj entitet ima \textit{Many-to-one} vezu s entitetom User preko atributa userId te \textit{One-to-many} vezu s entitetom Listings preko atributa scooterId.


\begin{longtblr}[
	label=none,
	entry=none
]{
	width = \textwidth,
	colspec={|X[10,l]|X[7, l]|X[15, l]|},
	rowhead = 1,
} %definicija širine tablice, širine stupaca, poravnanje i broja redaka naslova tablice
	\hline \SetCell[c=3]{c}{\textbf{Scooter}}	 \\ \hline[3pt]
	\SetCell{LightGreen}scooterId & INT	&  	jedinstveni identifikator romobila 	\\ \hline
	manufacturer	& VARCHAR & proizvođač romobila  	\\ \hline
	model & VARCHAR &  model romobila \\ \hline
	batteryCapacity & INT	& kapacitet baterije 		\\ \hline
	maxSpeed 	& INT &   maksimalna brzina	\\ \hline
	imagePath	& TEXT &  url slike 	\\ \hline
	maxRange	& FLOAT & maksimalni domet  	\\ \hline
	yearOfManufacture	& INT &   	godina proizvodnje\\ \hline
	additionalInformation	& TEXT &  dodatne informacije 	\\ \hline
	\SetCell{LightBlue}userId	& INT & jedinstveni identifikator korisnika (Ref: User.userId)	\\ \hline
	availability	& BOOLEAN &  dostupnost 	\\ \hline
	deleted	& BOOLEAN &  obrisan	\\ \hline
\end{longtblr}

\subsection{Listing}


Ovaj entitet sadrzava sve važne informacije o pojedinom oglasu. Sadrži atribute: listingId, currentAddress, returnAddress, returnByTime, pricePerKilometer, penaltyFee, scooterId, listingTime, notes, status i clientId. Ovaj entitet ima \textit{Many-to-one} vezu s entitetom Scooter preko atributa scooterId, \textit{Many-to-one} vezu s entitetom User preko atributa clientId te \textit{One-to-many} vezu s entitetom Transactions preko atributa listingId.


\begin{longtblr}[
	label=none,
	entry=none
]{
	width = \textwidth,
	colspec={|X[10,l]|X[7, l]|X[15, l]|},
	rowhead = 1,
} %definicija širine tablice, širine stupaca, poravnanje i broja redaka naslova tablice
	\hline \SetCell[c=3]{c}{\textbf{Listing}}	 \\ \hline[3pt]
	\SetCell{LightGreen}listingId & INT	&  jedinstveni identifikator oglasa	 	\\ \hline
	currentAddress	& VARCHAR & trenutna adresa  	\\ \hline
	returnAddress & VARCHAR	& adresa povratka 		\\ \hline
	returnByTime 	& TIMESTAMP & vrijeme vraćanja   	\\ \hline
	pricePerKilometer	& FLOAT &   cijena po kilometru	\\ \hline
	penaltyFee	& FLOAT &   	kaznena naknada\\ \hline
	\SetCell{LightBlue}scooterId	& INT &  jedinstveni identifikator romobila (Ref: Scooter.scooterId)	\\ \hline
	listingTime	& TIMESTAMP &   	vrijeme objave oglasa\\ \hline
	notes	& TEXT &  bilješke 	\\ \hline
	status	& ListingStatus &   	status oglasa (Available, Requested, Rented, Returned)\\ \hline
	clientId & INT	&  jedinstveni identifikator klijenta (Ref: User.userId)	\\ \hline
\end{longtblr}

\subsection{Review}


Ovaj entitet sadrzava sve važne informacije o pojedinom osvrtu. Sadrži atribute: reviewId, transactionId, reviewerUsername, renterUsername, stars, comment te reviewTime. Ovaj entitet ima \textit{One-to-one} vezu s entitetom User preko atributa reviewerUsername, \textit{One-to-one} vezu s entitetom User preko atributa renterUsername te \textit{One-to-one} vezu s entitetom Transaction preko atributa transactionId.


\begin{longtblr}[
	label=none,
	entry=none
]{
	width = \textwidth,
	colspec={|X[10,l]|X[7, l]|X[15, l]|},
	rowhead = 1,
} %definicija širine tablice, širine stupaca, poravnanje i broja redaka naslova tablice
	\hline \SetCell[c=3]{c}{\textbf{Review}}	 \\ \hline[3pt]
	\SetCell{LightGreen}reviewId & INT	&  jedinstveni identifikator osvrta	\\ \hline
	\SetCell{LightBlue}transactionId & INT	&  jedinstveni identifikator transakcije (Ref: Transaction.transactionId) 	\\ \hline
	\SetCell{LightBlue}reviewerUsername	& VARCHAR &  korisničko ime recenzenta (Ref: User.nickname) 	\\ \hline
	\SetCell{LightBlue}renterUsername & VARCHAR &  korisničko ime iznajmljivača (Ref: User.nickname) \\ \hline
	stars & INT	&  	broj zvjezdica/ocjena	\\ \hline
	comment	& TEXT &  komentar 	\\ \hline
	reviewTime	& TIMESTAMP &   vrijeme osvrta	\\ \hline
\end{longtblr}

\subsection{Transaction}


Ovaj entitet sadrzava sve važne informacije o pojedinoj transakciji. Sadrži atribute: transactionId, clientId, ownerId, kilometersTraveled, totalPrice, listingId, paymentTime te previousTransactionStatus. Ovaj entitet ima \textit{Many-to-one} vezu s entitetom Listing preko atributa listingId te \textit{One-to-one} vezu s entitetom Invoice preko atributa transactionId.


\begin{longtblr}[
	label=none,
	entry=none
]{
	width = \textwidth,
	colspec={|X[10,l]|X[7, l]|X[15, l]|},
	rowhead = 1,
} %definicija širine tablice, širine stupaca, poravnanje i broja redaka naslova tablice
	\hline \SetCell[c=3]{c}{\textbf{Transaction}}	 \\ \hline[3pt]
	\SetCell{LightGreen}transactionId & INT	&  jedinstveni identifikator transakcije	 	\\ \hline
	kilometersTraveled	& FLOAT &   broj prijeđenih kilometara	\\ \hline
	totalPrice	& FLOAT &   ukupna cijena	\\ \hline
	\SetCell{LightBlue}listingId	& INT &   jedinstveni identifikator oglasa (Ref: Listing.listingId)\\ \hline
	\SetCell{LightBlue}clientId	& INT &   jedinstveni identifikator klijenta (Ref: User.userId)\\ \hline
	\SetCell{LightBlue}ownerId	& INT &   jedinstveni identifikator vlasnika (Ref: User.userId)\\ \hline
	paymentTime	& TIMESTAMP &   vrijeme plaćanja	\\ \hline
	transactionStatus	& Transaction- Status &   status transakcije (Seen, Unseen)	\\ \hline
\end{longtblr}

\subsection{ChatSession}


Ovaj entitet sadrzava sve važne informacije o pojedinom razgovoru. Sadrži atribute: chatSessionId, user1, user2, startCommunicationTime te lastMessageTime. Ovaj entitet ima \textit{Many-to-one} vezu s entitetom User preko atributa user1, \textit{Many-to-one} vezu s entitetom User preko atributa user2 te \textit{One-to-many} vezu s entitetom Messages preko atributa sessionId.


\begin{longtblr}[
	label=none,
	entry=none
]{
	width = \textwidth,
	colspec={|X[10,l]|X[7, l]|X[15, l]|},
	rowhead = 1,
} %definicija širine tablice, širine stupaca, poravnanje i broja redaka naslova tablice
	\hline \SetCell[c=3]{c}{\textbf{ChatSession}}	 \\ \hline[3pt]
	\SetCell{LightGreen} chatSessionId & INT	&  	jedinstveni identifikator razgovora 	\\ \hline
	\SetCell{LightBlue}user1	& INT &   korisnik 1 (Ref: User.userId)	\\ \hline
	\SetCell{LightBlue}user2 & INT &  korisnik 2 (Ref: User.userId) \\ \hline
	startCommunicationTime 	& TIMESTAMP &   vrijeme početka komunikacije	\\ \hline
	lastMessageTime	& TIMESTAMP &   vrijeme zadnje poslane poruke	\\ \hline
\end{longtblr}

\subsection{Message}

Ovaj entitet sadrzava sve važne informacije o pojedinoj poruci. Sadrži atribute: messageId, senderUsername, sessionId, text, sentTime, MessageType te status. Ovaj entitet ima \textit{One-to-many} vezu s entitetom User preko atributa senderUsername te \textit{Many-to-one} vezu s entitetom ChatSession preko atributa sessionId.


\begin{longtblr}[
	label=none,
	entry=none
]{
	width = \textwidth,
	colspec={|X[10,l]|X[7, l]|X[15, l]|},
	rowhead = 1,
} %definicija širine tablice, širine stupaca, poravnanje i broja redaka naslova tablice
	\hline \SetCell[c=3]{c}{\textbf{Message}}	 \\ \hline[3pt]
	\SetCell{LightGreen}messageId & INT	&  	jedinstveni identifikator poruke 	\\ \hline
	\SetCell{LightBlue}senderUsername	& VARCHAR &   nadimak pošiljatelja (Ref: User.nickname)	\\ \hline
	\SetCell{LightBlue}chatSession & INT &  jedinstveni identifikator razgovora (Ref: ChatSession.chatId) \\ \hline
	\SetCell{LightBlue}userId & INT &  jedinstveni identifikator korisnika (Ref: User.userId) \\ \hline
	\SetCell{LightBlue}listingId & INT &  jedinstveni identifikator oglasa (Ref: Listing.listingId) \\ \hline
	text & TEXT	&  	tekst	\\ \hline
	sentTime 	& TIMESTAMP &   vrijeme slanja	\\ \hline
	status	& MessageStatus &   status poruke (read, unread)	\\ \hline
	messageType	& MessageType &   tip poruke (Regular Message, Request Message, Action Message)	\\ \hline
\end{longtblr}

\subsection{ImageChangeRequest}


Ovaj entitet sadrzava sve važne informacije o zahtjevu za promjenom slike. Sadrži atribute: imageId, requesterId, listingId, newImageUrl, oldImageUrl, complaintTime, additionalComments, status, approvalTime te rejectionReason. Ovaj entitet ima \textit{Many-to-one} vezu s entitetom User preko atributa requesterId te \textit{Many-to-one} vezu s entitetom Listing preko atributa listingId.


\begin{longtblr}[
	label=none,
	entry=none
]{
	width = \textwidth,
	colspec={|X[10,l]|X[7, l]|X[15, l]|},
	rowhead = 1,
} %definicija širine tablice, širine stupaca, poravnanje i broja redaka naslova tablice
	\hline \SetCell[c=3]{c}{\textbf{ImageChangeRequest}}	 \\ \hline[3pt]
	\SetCell{LightGreen}imageId & INT	&  	jedinstveni identifikator slike 	\\ \hline
	\SetCell{LightBlue}requesterId	& INT &   jedinstveni identifikator pošiljatelja (Ref: User.userId)	\\ \hline
	\SetCell{LightBlue}listingId & INT &  jedinstveni identifikator oglasa (Ref: Listing.listingId) \\ \hline
	oldImageUrl & VARCHAR	&  	url stare slike	\\ \hline
	newImageUrl & VARCHAR	&  	url nove slike	\\ \hline
	complaintTime 	& TIMESTAMP &   vrijeme žalbe	\\ \hline
	additionalComments	& TEXT &   dodatni komentari	\\ \hline
	status	& ImageChangeRequestStatus &  status zahtjeva (approved, rejected, requested, pending)	\\ \hline
	approvalTime	& TIMESTAMP &   vrijeme odobrenja	\\ \hline
	rejectionReason	& TEXT &   razlog odbitka	\\ \hline
\end{longtblr}



\subsection{Dijagram baze podataka}

\begin{figure} [H]
	
	\includegraphics[width=1\linewidth]{dijagrami/relacijskidijagram.png}
	\centering
	\caption{Prikaz dijagrama baze podataka}
	\label{fig:Prikaz dijagrama baze podataka}
\end{figure}

\subsection{Dijagram razreda}

Prvi isječak dijagrama razreda sadrži klasu User, enumeracije UserStatus i UserRole koje se nadovezuju na nju, klase PrivacySettings, Notification, SocialMedia, klasu Preference i odgovarajuću enumeraciju PreferenceUserLanguage te klasu Document povezanu s enumeracijom DocumentStatus.
\begin{figure} [H]

	\includegraphics[width=1\linewidth]{slike/ClassDiagram1.png}
	\centering
	\caption{Prikaz prvog isječka dijagrama razreda}
	\label{fig:Prikaz prvog isječka dijagrama razreda}
\end{figure}

Drugi isječak dijagrama razreda sadrži klase User, Scooter, klasu Listing s vlastitom enumeracijom ListingStatus, klasu ImageChangeRequest s vlastitom enumeracijom ImageChangeRequestStatus, klasu Review i klasu Transaction s vlastitom enumeracijom TransactionStatus.
\begin{figure} [H]

	\includegraphics[width=1\linewidth]{slike/ClassDiagram2.png}
	\centering
	\caption{Prikaz drugog isječka dijagrama razreda}
	\label{fig:Prikaz drugog isječka dijagrama razreda}
\end{figure}

Treći isječak dijagrama razreda sadrži klasu User, ChatSession povezanu s klasom Message koja ima dvije vlastite enumeracije, MessageType i MessageStatus.
\begin{figure} [H]

	\includegraphics[width=1\linewidth]{slike/ClassDiagram3.png}
	\centering
	\caption{Prikaz trećeg isječka dijagrama razreda}
	\label{fig:Prikaz trećeg isječka dijagrama razreda}
\end{figure}

\begin{figure} [H]

	\includegraphics[width=1\linewidth]{slike/ControllerDiagram.png}
	\centering
	\caption{Prikaz Controller dijagrama}
	\label{fig:Prikaz Controller dijagrama}
\end{figure}

\begin{figure} [H]

	\includegraphics[width=1\linewidth]{slike/ServiceDiagram.png}
	\centering
	\caption{Prikaz Service dijagrama}
	\label{fig:Prikaz Service dijagrama}
\end{figure}


\section{Dijagram stanja}

Dijagram stanja prikazuju objekte, njihova stanja i događaje kojima objekti prelaze iz jednog stanja u drugo. Slika prikazuje dijagram stanja za prijavljenog korisnika. Korisniku se inicijalno prikazuje početna stranica s trenutno objavljenim oglasima, ali ima i opciju pregleda svojih romobila, poruka i transakcija klikom na određenu opciju u navigacijskoj traci. Korisnik klikom na oglas na početnoj stranici može pregledati detalje o njemu i unajmiti ga. Ako korisnik već unajmljuje neki romobil, onda mu se ti romobili prikazuju s opcijom za vraćanje romobila. Na stranici za prikaz korisnikovih romobila postoje opcije za registraciju novih romobila, njihovo oglašavanje i pregled oglasa koje je korisnik već objavio. Klikom na ikonu korisničkog profila na navigacijskoj traci korisnik dobiva opciju pregleda, uređivanja i brisanja vlastitog profila.

\begin{figure} [H]

	\includegraphics[width=1\linewidth]{slike/StateDiagram.png}
	\centering
	\caption{Dijagram stanja - klijent}
	\label{fig:Dijagram stanja - klijent}
\end{figure}

\eject

\section{Dijagram aktivnosti}

Dijagramom aktivnosti prikazan je tijek prijave registriranog korisnika u sustav i njegova objava novog oglasa. Pravilnom prijavom u sustav korisnik dobiva mogućnost pregleda svojih registriranih romobila. Ako korisnik nema registriranih romobila, mora ih prvo registrirati popunjavanjem forme za registraciju. Nakon unosa podataka o romobilu podatci se spremaju u bazu podataka. Da bi objavio oglas korisnik prvo mora unijeti podatke o oglasu i oni se spremaju u bazu podataka. Uspješnim oglašavanjem korisnika se preusmjerava na početnu stranicu.

\begin{figure} [H]

	\includegraphics[width=1\linewidth]{slike/ActivityDiagram.png}
	\centering
	\caption{Dijagram aktivnosti - oglašavanje romobila}
	\label{fig:Dijagram aktivnosti - oglašavanje romobila}
\end{figure}

\eject

\section{Dijagram komponenti}



Dijagram komponenti prikazuje arhitekturu sustava i način na koji se komponente međusobno povezuju. Na slici 4.10 prikazan je UML dijagram komponenti sustava. Iz vanjske komponente Web preglednika se putem HTTP Request sučelja šalju klijentski zahtjevi prema aplikaciji. Codeblaze aplikacija sadrži dvije unutarnje komponente Frontend i Backend. Komunikacija između njih ostvarena je preko sučelja REST API. Frontend dio aplikacije izgrađen je pomoću React biblioteke, a sastoji se od React view koji služi za prikaz sučelja, React routera koji služi za upravljanje navigacijom i App.js koji služi kao središnja točka spajanja svih frontend elemenata. Na backendu, Rest Controller prima i obrađuje zahtjeve te ih prosljeđuje odgovarajućoj servisnoj komponenti. DTO komponenta služi za prijenos podataka između slojeva backenda. Komponenta repozitorija, koji koristi JPA za interakciju s bazom podataka, i domenske komponente, čine osnovu za upravljanje podacima. Preko sučelja PostgeSQL aplikacija se spaja na bazu podataka. 

\begin{figure} [H]
	\centering
	\includegraphics[width=1\linewidth]{dijagrami/DijagramKomponenti.png}
	\caption{Dijagram komponenti}
	\label{fig:Dijagram komponenti}
\end{figure}