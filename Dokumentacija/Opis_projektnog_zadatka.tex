\chapter{Opis projektnog zadatka}
		
		Cilj ovog projekta je razviti programsku podršku za stvaranje web aplikacije „Iznajmi romobil“ koja će korisnicima omogućiti da iznajme svoj električni romobil u periodima dana kada ga ne koriste. Aplikacija će korisnicima omogućavati brz i jednostavan pristup električnim romobilima dostupnima za najam kao i postavljanje ponude za iznajmljivanje svog romobila. Korisnici će moći za vrijeme kada su na poslu, kavi, treningu i slično odnosno kada ne koriste svoj električni romobil isti iznajmiti. Prilikom postavljanja ponude oni će odrediti gdje i do kada romobil mora biti vraćen. U slučaju da romobil ne bude vraćen na vrijeme, klijentu se naplaćuje naknada određena od strane iznajmljivača. Klijent će na temelju ponuđenih romobila i informacija o njima odabrati onaj koji je u tom trenutku dostupan i najviše odgovara njegovim potrebama. Funkcionalnosti aplikacije ovisiti će o vrsti korisnika. Aplikacija ima četiri vrste korisnika: neregistriranog korisnika, klijenta, iznajmljivača i administratora. S tim da isti korisnik može istovremeno biti i iznajmljivač i klijent.
		\newline 
		\newline
		Prilikom pokretanja aplikacije korisnicima se neovisno o tome jesu li prijavljeni ili ne prikazuje popis svih aktivnih ponuda romobila. Neregistrirani korisnici mogu pregledavati trenutno dostupne romobile i njihova cijene, ali ih ne mogu iznajmiti. Nakon što se prijave ili kreiraju novi korisnički račun, ponuđeni romobili im postaju dostupni za najam. Prilikom kreiranja novog računa korisnici moraju unijeti sljedeće podatke:
		
		\begin{packed_item}
			\item \textit{ime i prezime}
			\item \textit{email adresa}
			\item \textit{nadimak}
			\item \textit{broj kartice}
		\end{packed_item}
		
		Osim navedenog, korisnici prilikom registracije moraju dostaviti kopiju osobne iskaznice i potvrdu o nekažnjavanju. Nakon što su svi potrebni dokumenti dostavljeni, administrator pregledava dokumente te odobrava ili odbija registraciju. Dok administrator ne odobri registraciju, korisnik se ne može prijaviti u sustav. Kada mu administrator odobri registraciju, korisnik se prijavljuje u sustav. Ako korisnik prilikom unosa podataka za prijavu unese podatke koji ne odgovaraju nijednom registriranom korisniku u bazi, šalje mu se obavijest o neispravnosti podataka. BLOK U slučaju da administrator odbije zahtjev za registraciju zbog neispravnosti dostavljenih dokumenata, korisnik može ponovno predati dokumente. Svaki klijent u aplikaciji može pogledati svoj profil na kojem se nalaze njegovi osobni podaci te urediti isti odabirom opcije "Uredi profil".  Ako pri uređivanju profila dođe do unošenja podataka u neispravnom obliku, korisnik dobiva obavijest o neispravnosti. Nakon unosa promjena, korisnik mora odabrati opciju "Spremi promjene" kako bi potvrdio pohranjivanje promjena u bazu podataka. Administrator ima pravo blokirati korisnika odnosno zabraniti mu pristup aplikaciji odabirom opcije "Blokiraj korisnika". Za takvog se korisnika u bazu podataka upisuje da je blokiran te će mu pri sljedećoj prijavi biti onemogućen pristup sustavu. Iznajmljivač je korisnik koji postavlja svoj romobil u aplikaciju za iznajmljivanje. On prilikom registracije romobila unosi podatke o romobilu i postavlja sliku romobila koja dokazuje njegovo trenutno stanje. Prilikom postavljanja ponude za iznajmljivanje, iznajmljivač unosi trenutnu lokaciju romobila, lokaciju na koju želi da se romobil vrati, vrijeme do kada romobil mora biti vraćen, cijenu iznajmljivanja po prijeđenom kilometru te iznos novčane kazne u slučaju da romobil ne bude vraćen na vrijeme. Ako je neki romobil dostupan za iznajmljivanje, iznajmljivač oglas može objaviti i na nekoj društvenoj mreži odabirom opcije "Objavi na društvenu mrežu". Svaki iznajmljivač na svom profilu može pregledavati svoje registrirane romobile te ih brisati. Unutar aplikacije dostupna je mogućnost izmjenjivanja poruka korisnika. Klijent se, kada odabere romobil koji želi iznajmiti, javlja iznajmljivaču s porukom i zahtjevom za iznajmljivanje. Iznajmljivač pregledava zahtjeve za iznajmljivanje te tada može prihvaća ili odbija ponudu. Nakon što prihvati ponudu, klijentu se šalje obavijest da je iznajmljivanje odobreno i oglas se briše. Klijent prije pokretanja romobila provjerava odgovara li fotografija romobila njegovom stvarnom stanju. Ako ne odgovara, on ima mogućnost odabirom opcije "Zamijeni sliku" zamijeniti sliku romobila novom slikom i kratkim opisom o razlikama između nove i stare slike. Kada klijent zamijeni sliku, iznajmljivaču se šalje obavijest o zamjeni slika. On tada administratoru može poslati žalbu na zamjenu slika ukoliko smatra da klijentova slika ne odgovara stvarnom stanju romobila. Administrator nakon zaprimanja žalbe pregledava slike i odabire onu koja će se pohraniti u bazu. Nakon što administrator donese odluku, klijentu i iznajmljivaču se šalje obavijest o donesenoj odluci. Na kraju iznajmljivanja, klijent vraća romobil i u aplikaciji potvrđuje da ga je vratio. Nakon toga slijedi provjera je li romobil vraćen u pravo vrijeme te se izračunava cijena koju klijent mora platiti. Klijent i iznajmljivač dobivaju obavijest da je iznajmljivanje završeno i cijenu iznajmljivanja koju klijent treba platiti iznajmljivaču. Transakcija se izvršava i sprema u bazu podataka. Po završetku iznajmljivanja, iznajmljivač može ocijeniti klijenta i napisati komentar. Sve ocjene i komentari za pojedinog klijenta vidljivi su na njihovim profilima koje ostali korisnici mogu pregledavati. 
		
		
		
		
		
		
	