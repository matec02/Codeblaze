\chapter{Specifikacija programske potpore}
		
	\section{Funkcionalni zahtjevi}
			
			\textbf{\textit{dio 1. revizije}}\\
			
			\textit{Navesti \textbf{dionike} koji imaju \textbf{interes u ovom sustavu} ili  \textbf{su nositelji odgovornosti}. To su prije svega korisnici, ali i administratori sustava, naručitelji, razvojni tim.}\\
				
			\textit{Navesti \textbf{aktore} koji izravno \textbf{koriste} ili \textbf{komuniciraju sa sustavom}. Oni mogu imati inicijatorsku ulogu, tj. započinju određene procese u sustavu ili samo sudioničku ulogu, tj. obavljaju određeni posao. Za svakog aktora navesti funkcionalne zahtjeve koji se na njega odnose.}\\
			
			
			\noindent \textbf{Dionici:}
			
			\begin{packed_enum}
				
				\item Dionik 1
				\item Dionik 2				
				\item ...
				
			\end{packed_enum}
			
			\noindent \textbf{Aktori i njihovi funkcionalni zahtjevi:}
			
			
			\begin{packed_enum}
				\item  \underbar{Aktor 1 (inicijator) može:}
				
				\begin{packed_enum}
					
					\item funkcionalnost 1
					\item funkcionalnost 2
					\begin{packed_enum}
						
						\item  podfunkcionalnost 1 
						\item  podfunkcionalnost 2
				
					\end{packed_enum}
					\item  funkcionalnost 3
					
				\end{packed_enum}
			
				\item  \underbar{Aktor 2 (sudionik) može:}
				
				\begin{packed_enum}
					
					\item funkcionalnost 1
					\item funkcionalnost 2
					
				\end{packed_enum}
			\end{packed_enum}
			
			\eject 
			
			
				
			\subsection{Obrasci uporabe}
				
				\textbf{\textit{dio 1. revizije}}
				
				\subsubsection{Opis obrazaca uporabe}
					
					
					\noindent \underbar{\textbf{UC1 - Pregledaj romobile}}
					\begin{packed_item}
	
						\item \textbf{Glavni sudionik: }Korisnik
						\item  \textbf{Cilj:} Pregledati romobile dostupne za iznajmljivanje
						\item  \textbf{Sudionici:} Baza podataka
						\item  \textbf{Preduvjet:} -
						\item  \textbf{Opis osnovnog tijeka:}
						
						\item[] \begin{packed_enum}
	
							\item Učitava se početna stranica aplikacije
							\item Prikazuje se ponuda romobila dostupnih za iznajmljivanje
							\item Sudionik pregledava dostupne romobile i informacije o njima
						
						\end{packed_enum}
					\end{packed_item}
						
							\noindent \underbar{\textbf{UC2 - Registriraj korisnika}}
						\begin{packed_item}
							
							\item \textbf{Glavni sudionik: } Neregistrirani korisnik
							\item  \textbf{Cilj:} Napraviti korisnički račun kojim se pristupa sustavu
							\item  \textbf{Sudionici:} Administrator, baza podataka
							\item  \textbf{Preduvjet:} -
							\item  \textbf{Opis osnovnog tijeka:}
							
							\item[] \begin{packed_enum}
								
								\item Neregistrirani korisnik odabire opciju „Registriraj se“ 
								\item Neregistriranom korisniku prikazuje se stranica za registraciju 
								\item Neregistrirani korisnik unosi podatke za registraciju 
								\item Unesena kopija osobne iskaznice i potvrda o nekažnjavanju šalju se administratoru na pregled 
								\item Stvara se novi korisnički račun čiji se podatci pohranjuju u bazu podataka 
								\item Korisnika se preusmjerava na stranicu za prijavu u sustav 
								
							\end{packed_enum}
							
							\item  \textbf{Opis mogućih odstupanja:}
							
							\item[] \begin{packed_item}
								
								\item[3.a] Unos podataka u nedozvoljenom formatu ili unos već zauzetog nadimka ili e-mail adrese 
								\item[] \begin{packed_enum}
									
									\item Korisnik dobiva odgovarajuću obavijest o neispravnosti podataka 
									\item Korisnik mijenja potrebne podatke i završava s unosom ili odustaje od registracije 
									
								\end{packed_enum}
								
								
							\end{packed_item}
						\end{packed_item}
						\noindent \underbar{\textbf{UC3 - Prijavi korisnika}}
						\begin{packed_item}
							
							\item \textbf{Glavni sudionik: } Klijent, iznajmljivač
							\item  \textbf{Cilj:} Dobiti pristup korisničkim funkcijama
							\item  \textbf{Sudionici:} Baza podataka
							\item  \textbf{Preduvjet:} Korisnik je registriran
							\item  \textbf{Opis osnovnog tijeka:}
							
							\item[] \begin{packed_enum}
								
								\item Korisnik odabire opciju „Prijavi se“ 
								\item Korisniku se prikazuje stranica za prijavu  
								\item Korisnik unosi podatke za prijavu 
								\item Prijava je odobrena i korisnik dobiva pristup korisničkim funkcijama 
								 
								
							\end{packed_enum}
							
							\item  \textbf{Opis mogućih odstupanja:}
							
							\item[] \begin{packed_item}
								
								\item[3.a] Unos podataka koji ne odgovaraju nijednom registriranom korisniku u bazi podataka  
								\item[] \begin{packed_enum}
									
									\item Korisnik dobiva odgovarajuću obavijest o neispravnosti podataka 
									\item Korisnik mijenja potrebne podatke i završava s unosom ili odustaje od prijave 
									
								\end{packed_enum}
								\item[4.a] Prijava nije odobrena jer korisnik čeka na odobrenje registracije    
								\item[] \begin{packed_enum}
									
									\item Korisnik dobiva obavijest da je njegov zahtjev za registraciju na čekanju  
									\item Ako se zahtjev za registraciju odobri, korisnik se uspješno prijavljuje u sustav, a ako se zahtjev za registraciju odbije, korisnika se preusmjerava na stranicu gdje može ponovno predati dokumentaciju za registraciju na provjeru 
									
								\end{packed_enum}
									\item[4.b] Prijava nije odobrena jer je korisnik blokiran i nema pristup sustavu 
								\item[] \begin{packed_enum}
									
									\item Korisnik dobiva obavijest da je blokiran i nema više pristup sustavu   
									
									
								\end{packed_enum}
								
								
							\end{packed_item}
						\end{packed_item}
						\noindent \underbar{\textbf{UC4 - Ponovno predaj dokumentaciju}}
						\begin{packed_item}
							
							\item \textbf{Glavni sudionik: }Neregistrirani korisnik
							\item  \textbf{Cilj:} Ponovno predati kopiju osobne iskaznice i potvrdu o nekažnjavanju na pregled
							\item  \textbf{Sudionici:} Administrator, baza podataka
							\item  \textbf{Preduvjet:} Prethodno odbijen zahtjev za registraciju
							\item  \textbf{Opis osnovnog tijeka:}
							
							\item[] \begin{packed_enum}
								
								\item Korisnik unosi novu dokumentaciju za registraciju u sustav 
								\item Dokumentacija se šalje administratoru na pregled 
								\item Korisnika se preusmjerava na stranicu za prijavu gdje se pokušava ponovno prijaviti u sustav 
								
								
							\end{packed_enum}
							
							\item  \textbf{Opis mogućih odstupanja:}
							
							\item[] \begin{packed_item}
								
								\item[1.a] Unos dokumentacije u nedozvoljenom formatu 
								\item[] \begin{packed_enum}
									
									\item Korisnik dobiva odgovarajuću obavijest o neispravnosti podataka  
									\item Korisnik unosi ispravnu dokumentaciju i završava s unosom ili odustaje od ponovne predaje dokumentacije  
									
								\end{packed_enum}
								
								
							\end{packed_item}
						\end{packed_item}
						\noindent \underbar{\textbf{UC5 - Pregledaj profil}}
						\begin{packed_item}
							
							\item \textbf{Glavni sudionik: } Klijent, iznajmljivač
							\item  \textbf{Cilj:} Pregledati korisničke podatke svog profila
							\item  \textbf{Sudionici:} Baza podataka
							\item  \textbf{Preduvjet:} Korisnik je prijavljen
							\item  \textbf{Opis osnovnog tijeka:}
							
							\item[] \begin{packed_enum}
								
								\item Korisnik odabire opciju „Moj profil“ 
								\item Korisniku se prikazuje stranica vlastitog profila i svi njegovi korisnički podatci   
								\item Korisnik pregledava informacije o svom profilu 
							 
							\end{packed_enum}
						\end{packed_item}
						\noindent \underbar{\textbf{UC6 - Uredi profil}}
						\begin{packed_item}
							
							\item \textbf{Glavni sudionik: }Klijent, iznajmljivač
							\item  \textbf{Cilj:}Promijeniti korisničke podatke i odlučiti koji će od njih biti javni, a koji privatni
							\item  \textbf{Sudionici:} Baza podataka
							\item  \textbf{Preduvjet:} Korisnik je prijavljen
							\item  \textbf{Opis osnovnog tijeka:}
							
							\item[] \begin{packed_enum}
								
								\item Korisnik odabire opciju „Uredi profil“  
								\item Korisnik mijenja svoje korisničke podatke i njihovu dostupnost 
								\item Korisnik potvrđuje promjene odabirom opcije „Spremi promjene“ 
								\item Baza podataka se ažurira 
								
								
							\end{packed_enum}
							
							\item  \textbf{Opis mogućih odstupanja:}
							
							\item[] \begin{packed_item}
								
								\item[2.a] Unos podataka u nedozvoljenom formatu ili unos već zauzetog nadimka ili e-mail adrese 
								\item[] \begin{packed_enum}
									
									\item Korisnik dobiva odgovarajuću obavijest o neispravnosti podataka 
									\item Korisnik mijenja potrebne podatke i završava s unosom ili odustaje od promjene 
									
								\end{packed_enum}
								\item[3.a] Korisnik ne potvrdi promjenu odabirom opcije „Spremi promjene“    
								\item[] \begin{packed_enum}
									
									\item Korisnik dobiva obavijest da nije spremio podatke prije izlaska iz prozora za promjenu podataka   
									\item Korisnik sprema promjene 
									
								\end{packed_enum}
								
								
								
							\end{packed_item}
						\end{packed_item}
								
						
				
					
				\subsubsection{Dijagrami obrazaca uporabe}
					
					\textit{Prikazati odnos aktora i obrazaca uporabe odgovarajućim UML dijagramom. Nije nužno nacrtati sve na jednom dijagramu. Modelirati po razinama apstrakcije i skupovima srodnih funkcionalnosti.}
				\eject		
				
			\subsection{Sekvencijski dijagrami}
				
				\textbf{\textit{dio 1. revizije}}\\
				
				\textit{Nacrtati sekvencijske dijagrame koji modeliraju najvažnije dijelove sustava (max. 4 dijagrama). Ukoliko postoji nedoumica oko odabira, razjasniti s asistentom. Uz svaki dijagram napisati detaljni opis dijagrama.}
				\eject
	
		\section{Ostali zahtjevi}
		
			\textbf{\textit{dio 1. revizije}}\\
		 
			 \textit{Nefunkcionalni zahtjevi i zahtjevi domene primjene dopunjuju funkcionalne zahtjeve. Oni opisuju \textbf{kako se sustav treba ponašati} i koja \textbf{ograničenja} treba poštivati (performanse, korisničko iskustvo, pouzdanost, standardi kvalitete, sigurnost...). Primjeri takvih zahtjeva u Vašem projektu mogu biti: podržani jezici korisničkog sučelja, vrijeme odziva, najveći mogući podržani broj korisnika, podržane web/mobilne platforme, razina zaštite (protokoli komunikacije, kriptiranje...)... Svaki takav zahtjev potrebno je navesti u jednoj ili dvije rečenice.}
			 
			 
			 
	