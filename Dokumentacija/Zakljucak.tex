\chapter{Zaključak i budući rad}
		
		\textbf{\textit{dio 2. revizije}}\\
		
		 \textit{U ovom poglavlju potrebno je napisati osvrt na vrijeme izrade projektnog zadatka, koji su tehnički izazovi prepoznati, jesu li riješeni ili kako bi mogli biti riješeni, koja su znanja stečena pri izradi projekta, koja bi znanja bila posebno potrebna za brže i kvalitetnije ostvarenje projekta i koje bi bile perspektive za nastavak rada u projektnoj grupi.}
		
		 \textit{Potrebno je točno popisati funkcionalnosti koje nisu implementirane u ostvarenoj aplikaciji.}
		
		
		Projektni zadatak naše grupe bio je izrada web aplikacije koja će korisnicima omogućiti iznajmljivanje vlastitih romobila kao i unajmljivanje romobila drugih korisnika. \newline
		Za izradu aplikacije imali smo otprilike 19 tjedana, a ona je podijeljena u dvije faze.
		U prvoj je fazi naglasak bio na izradi dokumentacije i implementaciji generičkih funkcionalnosti dok je u drugoj fazi naglasak bio na implementaciji projekta, ispitivanju implementacije izrađene aplikacije te puštanju aplikacije u pogon.
		Nakon uvodne vježbe s asistentom i međusobnog upoznavanja, započeli smo s diskusijom tehnologija koje ćemo koristiti i planiranjem faza izrade aplikacije. Nakon toga započeli smo s intenzivnim radom na izradi dokumentacije; obrazaca uporabe, dijagrama obrazaca uporabe, sekvencijskih dijagrama i arhitekture sustava.
		Kako smo se svi prvi put susreli s ovim oblikom izrade projekta, imali smo dosta nedoumica oko izvedbe, a posebice oko dijagrama koje smo tek tada usvajali na nastavi. Kada smo ispravno definirali obrasce uporabe i iste uspješno prikazali dijagramima, razjasnili smo brojne nedoumice oko funkcionalnosti aplikacije koje smo prethodno imali. Nakon toga, dio tima radio je na izradi ostatka dokumentacije, a dio je počeo raditi na implementaciji generičkih funkcionalnosti. Do prve predaje uspješno smo napisali svu potrebnu dokumentaciju i implementirali sve do tada potrebne funkcionalnosti. \newline
		Druga faza našeg rada na aplikaciji započela je nakon dodjele bodova kada smo detaljnije razradili plan za daljnji razvoj aplikacije. Kako je u toj fazi bilo više posla oko same implementacije, a manje oko dokumentacije, odlučili smo se podijeliti u pod timove. Svaki pod tim imao je članove koji su  radili na frontendu i one koji su radili na backendu aplikacije. Kako bi bili što efikasniji, podijelili smo zadatke tako da je svaki pod tim morao napraviti određenu funkcionalnost aplikacije. Naravno, kada netko naišao na neki problem prilikom pisanja koda, međusobno smo si pomagali. Tijekom cijelog rada na projektu često smo se nalazili na sastancima kako bismo prodiskutirali nedoumice i izvijestili ostatak tima o napretku pojedinog pod tima. Kao i većina timova, posebice onih s manjkom iskustva poput nas,  i mi smo se susreli s brojnim izazovima i poteškoćama rada u timu. Naša raspodjela u pod timove je bila dobra, ali zadatke nismo baš najbolje podijelili pa se događalo da jedan podtim ovisi o drugom odnosno da ne može raditi na svom zadatku dok drugi tim ne dovrši. Time smo si malo otežali posao jer bi bilo trenutaka kada su neki imali jako puno posla dok drugi nisu imali šta za raditi. 
		\eject 